\documentclass[11pt,a4paper,twoside,openany]{book}
\usepackage[utf8]{inputenc}
\usepackage[T1]{fontenc}
\usepackage{fullpage}

\begin{document}
\chapter*{Francisca}
Caminhando, ele tirou os olhos do documento que carregava e procurou o cliente, a quem deveria entregá-lo; no placar piscava uma senha: 478. Não o avistando, prontamente chamou seu nome com aquela voz treinada e o tom de impaciência de quem faz isso inúmeras vezes ao dia. Automático, alongando as vogais na tentativa de vencer o barulho do salão cheio, chamou: "João Pedro!"

"Francisca. Meu nome é Francisca", disse, se debruçando no balcão, ao atendente do outro lado.

Ela sabia que seria mais um constrangimento, aquela ida ao cartório. Sempre que precisava apresentar seus documentos se sentia constrangida, mas dessa vez não havia alternativa, sua irmã precisava dessa procuração para resolver seus assuntos bancários no interior, de onde tinha vindo.

Há anos que vivia na capital, há anos que já havia assumido sua posição no mundo, com muito custo; custo de preconceito e de dor. Todos os seus amigos a conheciam por Francisca, alguns amores no calor das noites frias tinha lhe chamado de Chicaç ela nunca achou ruim. Ruim era ser chamada de João Pedro.

Ainda não tinha conseguido o direito de trocar seus documentos; sabia de uma advogada que tinha já uns casos desses, umx amigx havia lhe dado o contato, mas considerando todos os documentos que teria que levantar, além do fato da demora sabida para o processo como um todo, duvidava que estaria viva quando a resposta, favorável ou não, lhe coubesse. Preferia evitar os ambientes em que fosse obrigada a expor sua \emph{condição}.

Seus pais não tinha  sabido muito bem o que era transexualidade, até ela dizer para eles, numa tarde de café, na mesa da cozinha, que lá na cidade ninguém a conhecia por João Pedro; que ela se sentia, se via, se fazia Francisca. Que queria mudar de sexo, tirar seu pinto e colocar no lugar uma vagina.

O pai, sem muita instrução, mas com coração que compensava qualquer ignorância, perguntou atencioso e espantado, se isso era mesmo possível; ela disse que sim, com poças nos olhos, sorrindo afeto. A mãe, relutante, perguntava se ela tinha certeza --- e de onde tinha que ela tinha tirado o nome de Francisca.

Ali, estava vestida de João Pedro, como fazia sempre que visitava os pais, o que faria pela última vez naquele dia. Calça preta, camisa de botões no peito reto, cabelos presos, sapatos.

Tinha trazido um vestido na mala e no dia seguinte já se apresentou Francisca-Francisca para o café da manhã.

Apesar do estranhamento inicial, visível nos primeiros olhares deles, ela exalava normalidade, que contagiou os pais e antes do almoço, João Pedro era Francisca na casa onde João Pedro nascera, na que Francisca até então se sentia como uma estranha.

Petrônio e Elsa, desde muito cedo sabiam que João Pedro era diferente. Ele sempre foi mais recatado e sensível que os meninos do bairro, nunca se escondeu de uma briga, mas não ficava pela rua puxando cabelos das meninas, ou procurando minhocas para assustá-las. Enquanto os meninos arquitetavam guerras com os vizinhos da rua de baixo, construindo castelos de estratégias mirabolantes, ele desinteressado, sentava junto às meninas, discutindo se Maria realmente já tinha beijado Fernando, como dizia que tinha.

Se interessava pelo mundo delas e sempre teve mais amiguinhas que amiguinhos. Mas a mãe nunca tomou isso por agrave, já que o Petrônio recebia o filho e o ouvia, quando João Pedro, sentado em seu colo, contava das tardes que passava na casa da Júlia ou da Mariana. Por que ela havia de se importar? O filho era feliz e os olhos do pai brilhavam com os da criança, sempre.

A irmã mais nova também gostava da companhia, sempre que a diferença de idade permitia. Joana, depois de uma certa idade, começou a perguntar a João Pedro sobre namoradas, trazia recados das amiguinhas do bairro que se interessavam por ele, e ele, sempre muito atencioso, dizia que não sabia exatamente o que fazer com essas cartinhas. Ficava sem graça, se fazia de desentendido, desconversava e o assunto, eventualmente, morria. Vez ou outra, uma menina mais insistente se impunha, eles trocavam beijinhos, ficavam ambos com vergonha --- talvez as meninas que o beijavam já sentissem que o segundo beijo não viria. João Pedro nunca teve fama disso ou daquilo, mas nunca teve namorada também.

Quando tinha dezessete anos, uma tia o convidou para ir à capital passar férias; nunca mais voltou. Terminou o colégio e começou a faculdade por lá mesmo, trabalhando de dia e estudando à noite, para se sustentar --- porque a tia, que gostava muito dele, "nâo ia sustentar marmanjo nenhum". Assim, foi aprendendo a viver na cidade, entre uma aula e outra, um trabalho e outro.

Fez alguns amigos que eram tão confusos sobre sua sexualidade quanto ele. Não se achava homossexual, também não se via hétero e durante muitos anos se marginalizou e se culpou por algo que nem sabia exatamente o que era.

Ao terminar a faculdade de administração, trabalhando num \emph{call center}, depois de muito se ver entre homens e mulheres, um dia, num bar, encontrou Beta. Linda, de cabelos longos, pretos, seios que enchiam seu decote voluptuoso e uma feminilidade sutil nas mãos grandes e bem cuidadas. Era amiga de um colega de trabalho. Começaram a conversar e, quanto mais conversavam, mais ele entendia que a Beta não era exatamente uma mulher --- e isso o fascinava. Continuaram saindo e ele continuou a explorar esse interesse, se perguntava o que era tão interessante na Beta, porque se sentia atraído por ela.

Um dia, na casa da Beta, experimentou um vestido pela primeira vez. De imediato, no espelho, elx não via mais João, nem Pedro, elx via a si mesmx, num vestido florido, verde, com alcinhas sobre os peitos se mvolume. Via suas pernas peludas e seus braços magros dentro de um vestido delicado. Seu corpo, emoldurado pelos limites do espelho, com o quarto amplo e iluminado de fundo, exibia para seus olhos, pela primeira vez, um transexual. Via-se a si mesmo, no espelho, como se via há muito nos olhos de sua alma, sem saber exatamente o que faltava para se ver manifesto como era. Finalmente a realidade do mundo dos outros se encontrava com a sua.

Ao ver que lágrimas escorriam furtivas de seus olhos no espelho, começou a se perguntar como seria daqui pra em diante e na conversa que se seguiu, que durou quase o dia todo, elx saiu dali vestindo o vestido verde, símbolo de si mesmx. Beta insistira.

Naquela noite não dormiu, extasiadx que estava com sua descoberta. Em casa, sentava e levantava da cadeira, cozinhava e e se olhava no espelho do banheiro, de vestido, alegre, incontinente, dormiu de cansaço no sofá, depois de muito fazer vestindo seu recém-descoberto conforto consigo mesmx.

Acordou no dia seguinte, segunda-feira, e se viu num dilema: ir ou não de vestido ao trabalho.

Convenceu-se de que aquele vestido não era roupa para o ofício e vestiu sua roupa séria de João. Ao ver-se no espelho, algo se apagou. No sorriso que dava aos colegas de trabalho faltava alma, faltava elx. Por dias chegou em casa e colocou o mesmo vestido verde florido, e se reacendia.

Semanas depois, já dilacerada pela ambiguidade de ser e não ser João, chorando, procurava alguma reza nas prateleiras da sala. Porque se Deus existia havia de lhe ajudar quando, não enfrentando um dilema do mundo, mas um dilema de si mesmx, procurava a assistência Dele, a quem raramente pedia qualquer coisa.

Encontrou um panfletinho que lhe entregaram na volta do trabalho, uns dias atrás. Ele, que não sabia o que lhe davam, sem atenção, tomou o papel e de relance viu a imagem de um santo qualquer. Não sendo religioso, voltou sua atenção para sua condução, que se aproximava e viu que precisava apertar o passo para não a perder. Apressadamente, colocou o papel dentro do livro que trazia e o esqueceu.

Encontrou o livro na poltrona; Oração de Francisco de Assis, dizia o panfleto, de um lado. Ao terminar de ler, pela décima vez, já em voz alta, sabia que seria a partir dali Francisca.

No dia seguinte, faltou ao trabalho, deu desculpa qualquer e foi faze  compras em seu vestido verde florido. Pela primeira vez entrou numa loja feminina, bastante sem jeito, mas totalmente inconsciente dos olhares alheios tamanha a sua alegria em se ver como se era. Entre essa e aquela sapatilha experimentada sorria e admirava seus pés grosseiros semiocultos pela delicadeza dos padrões que sempre se encerravam na altura do peito do pé. Passou pela sua cabeça provar um salto, mas apesar de curiosx, esse era um desafio impensável. \emph{"E tenho mesmo é que pensar no dia a dia"}, dizia par asi mesma, \emph{"numa roupa que possa usar no trabalho no dia seguinte"}. No departamento feminino comprou peças que achava sóbrias e respeitadoras para seu ambiente de trabalho e voltou para casa feliz, calçando sua ssapatilhas novas.

No dia seguinte, acordou cedo, fez a barba e colocou sua roupa nova. Saiu de casa como se tivesse visto o sol pela primeira vez ao cruzar a porta da rua. Para ela, era como se aquele fosse o primeiro e mais lindo dia de primavera de toda a criação; sentia-se flutuar e sorria por dentro, ainda que sua aparência fosse contida e séria.

Ao chegar no trabalho, foi recebida com olhares de espanto e de piada. Alguns dos que a viam todos os dias, nos primeiros segundos não a reconheciam, depois não sabia mo que dizer. Não sabia como entender os olhares que recebia dos amigos, que até semana passada o tratavam como igual. Ao passar, ouvia cochichos e risadinhas abafadas --- de um, entendeu que dava para ver os pelos de João Pedro dentro da meia-calça. Durante o dia, teve que chamar atenção de um subordinado e esse achou que não lhe devia respeito, que o João estava louco, que "tinha tirado o dia para mangar dos colegas... ou então que era viado mesmo".

No dia seguinte, ao chegar, foi chamada pelo chefe. Não podia continuar vindo trabalhar assim, ele disse. ao que ela perguntou, "como deveria então"? "Como homem", foi a resposta que recebeu. Ela, com muito pesar no coração, vislumbrando pela primeira vez como seria, daqui em diante, ser quem era (mas que outra escolha tinha?), olhou profundamente para seu chefe e disse "Lúcio, sou homem, mas não como você. Tenho sim, o mesmo corpo, mas me vejo assim, como você me vê agora. Não posso deixar de vir trabalhar como me vejo; me visto mais discreta que algumas das outras mulheres que vejo por aqui; não estou desrespeitando ninguém". E foi informado que se viesse assim no dia seguinte, seria demitido.

Nem se deu ao trabalho de sair da cama no dia seguinte. Ligou para o Lúcio. Quando ele atendeu ouviu apenas "peço as contas". Depois de tanto tempo se procurando, não abriria mão de ser quem era.

Durante algum tempo aguentou o desemprego. Aproveitou para pegar dicas com as amigas da Beta, começou a buscar formas de se fazer mais feminina. Uma vez até depilou as pernas com cera, mas não repetiria a dose tão cedo; se contentaria com aqueles depiladores de farmácia. Comprou esmaltes. Começou a cultivar os cabelos como quem cultiva flores. Nas noites, fez amizades e amores; continuava a entreter amores de ambos os sexos, ora gostando de ser quem puxava pelo cabelo, ora gozando ao ter seus cabelos puxados.

Depois de muitas entrevistas, o dinheiro começou a apertar. Com currículo no nome de Francisca Lemes Souza (tinha atualizado sua foto, que agora apresentava um simpático sorriso nos lábios vermelhos sob olhos delineados) buscava vagas na função que tinha antes da Francisca vir ao mundo. Muitas entrevistas e nada de emprego. Via na cara dos entrevistadores, logo que os conhecia, que não era sua competência que a colocava em xeque. Entrevista por entrevista, dentro de si, um amargor e um desamparo cresciam.

Pediu dinheiro emprestado aos pais e à irmã, decidiu que faria curso de esteticista, dica de uma das amigas da Beta. Logo na primeira aula se sentiu acolhida, se apresentava como Francisca e era respeitada como Francisca. Riam-se o dia inteiro, ela e as colegas. Ao final, fez também o curso de cabeleireira. Arrumou um emprego no bairro de periferia em que morava. Seguiu sua vida.

Nunca mais foi a mesma. As tentativas e as frustrações de se colocar no mundo como Francisca, aos poucos a fizeram notar os olhares de zombaria, os cochichos na fila do supermercado ou na lotérica. De salto, era maior que muito homem, se impunha, era alta, elegante, mas tinha as mãos grandes. O tratamento hormonal que começou há alguns anos afinava a barba e a voz, mas não fazia as mãos diminuírem. Começou a juntar dinheiro para implantes. Queria seios como lembrava serem os da Beta, a quem já não via mais --- mudara-se para algum outro lugar, "o dinheiro era melhor", ela disse ao se despedir.

Uma noite, num bar, um homem de terno, já meio bêbado, chegou para conversar de olhos fixos em seu decote cheio. Ela, que estava com amigxs e não muito sóbria, deu conversa. Acabaram na casa dela, tiveram uma noite de calores que iam e que vinham, ora dela, ora dele. Dormiram exaustos e suados, um ao lado do outro.

Quando ela acordou, ele já não estava. Viu, sobre o criado-mudo, duzentas moedas. Deitada e semiacordada, sentiu-se afundar. Não era dessas, não sabia nada além do primeiro nome dele, Alberto, e não poderia tomar satisfação. Deixou o dinheiro onde estava, sem tocá-lo, como se fosse carniça.

Dias o dinheiro ficou ali, sendo ignorado, juntando poeira. Durante a faxina semanal, foi solenemente ignorado e eficientemente evitado. Numa quarta à tarde, no intervalo do salão, enquanto preparava o almoço da semana, o gás acabou e o único dinheiro da casa era o do criado-mudo. Comprou o gás e seguiu sua vida. Eventualmente, o resto do dinheiro encontrou caminho até sua bolsa e sumiu nas necessidades de Francisca.

Duzentas moedas. Era mais do que fazia em uma semana de salão, depois de todos os descontos. Naturalmente, começou a frequentar lugares onde era bem aceita em sua exuberância peculiar. Nesses lugares, homens se aproximavam dela; uns puxavam conversa como quem queria conhecer Francisca, outros eram mais pragmáticos e perguntavam de cara "quanto você cobra?"

No primeiro descaramento, depois de olhar fixamente nos olhos de quem perguntava enquanto engolia sua covardia com um gole de cuba-libre, ouviu-se dizer: duzentas moedas.

Descobriu prazer nas aventuras noturnas. Morria de medo de ser descoberta pelas amigas do bairro ou do salão, onde continuava a trabalhar durante o dia. Com o dinheiro que ganhava se cuidava mais: depilação a laser, que a deixava muito mais delicada do que as lâminas e doía muito menos do que as ceras; cuidava dos cabelos em salões no centro da cidade, comprava roupas para suas noites.

Aos poucos, suas amigas de salão começaram a perceber as mudanças e perguntavam quem era o benfeitor que a estava tratando assim tão bem. Por omissão de Francisca, tentando desconversar como fazia quando criança, o Benfeitor aparecia rotineiramente entre um corte e outro. Ela mesma começou a dizer que "encontraria o Benfeitor" naquela noite, envaidecida pela admiração que conseguira entre as amigas de ofício.

Tinha vinte e oito anos, já não tinha mais barba, estava completamente acostumada aos implantes, frequentava academia para manter a forma; já tinha feito duas lipos, guardava dinheiro para o implante de glúteos, quando uma colega de trabalho descobriu, por acidente, que Francisca se prostituía. O respeito se converteu em nojo, suas colegas de salão a expulsaram de lá quando ela admitiu que fazia dinheiro com prazer.

Novamente, viu seus amigxs mudarem por conta de suas escolhas; novamente mudava de vagão na vida e via ao seu lado outros passageiros. Tornando sua ocupação principal aquela que lhe dava mais prazer, fez dinheiro e começou a enfrentar a solitude da vida novamente.

Com tempo, deu entrada num apartamento em região respeitável da cidade.

Os porteiros do prédio em que morava a chamavam de Dona Francisca, mas tinham, no fundo dos olhos, aquela reprovação mundana de quem julga a quantidade de homens que subiam ao seu apartamento.

Os anos se passaram e nunca encontrou, na maturidade, um amor, como já não havia encontrado na juventude.

Entre seus clientes, encontrou ombros e conversas, dividia parte de sua intimidade com os mais antigos, mas esses, que eram mais atenciosos, ironicamente, eram casados. Homens probos aos olhos do mundo, que se desnudavam na cama de Francisca, pais de família que se expunham crianças carentes de afago ou desesperados por alguém que soubesse mandar-lhes e amansar-lhes os instintos.

Sozinha, viajou e viu lugares, conheceu pessoas, preferia a companhia daquelas que a respeitavam inteiramente --- infelizmente essas não eram pessoas tão fáceis de se achar. Nunca foi demasiadamente amada, nunca foi para ninguém uma Beta. Era completamente feliz em ter se achado, em ter se descoberto Francisca. Noites de solidão a tinham ensinado a cuidar de si mesma, tinham-na feito mais forte.

Enfim, já madura, financeiramente segura, deu entrada no processo de reconhecimento judicial de sua condição, no mesmo ano em que seu pai morreu.

Francisca, que vivia na cidade e raramente visitava a família estendida, encontrou com tios e primos depois de muito tempo, vestida de preto, de salto e de decote. Não pode manifestar, durante aqueles dias, todo o amor que sentia pelo pai que se despedia, porque conflitava dentro de si um ódio pelo mundo que nunca a aceitaria como seu pai a aceitou. A força que recebia do amor incondicional dele era maior do que qualquer coragem que jamais teve. E os olhares dos parentes-estranhos, que a encaravam e cochichavam sobre o caixão sendo depositado na terra, a queimavam e faziam desse momento um dos piores da sua vida. Odiava a todos, amava seu pai, abraçava-se com mãe e irmã; chorava, mas os olhos ferviam.

Vendeu o apartamento na cidade e mudou-se novamente para o interior. O dinheiro que tinha guardado e o do apartamento, aplicados e administrados, poderiam suportá-la de forma humilde até o fim da vida. E ela haveria de encontrar atividade na cidade onde nascera. O importante agora era cuidar da mãe, que sozinha, definha a olhos vistos, morria de tristeza e solidão como quem escorre pelos orifícios do mundo.

Foram dois anos de cuidados e de despedida, até que sua mãe descansou. Como quando seu pai morrera, cuidou de todos os custos e burocracias de um enterro. Como quando seu pai morrera, encontrou parentes-estranhos e mais uma vez os odiou enquanto chorava a falta de alguém que não seria possível substituir nessa vida.

No final da quarta-feira, sua irmã se despedia com a família, partindo de volta para a cidade onde moravam. Ao fechar a porta da casa que fora de Petrônio e Elsa, agora vazia, ecoando o barulho do motor da geladeira, ela desabou sozinha, toda a vida que tivera até ali não a havia preparado para tanta solidão. Naquela noite, dormiu ao pé da porta da sala, depois de chorar até desacordar.

Acordou sem vontade de ver pessoas, sem esperança de, em algum ponto a partir dali, encontrar conforto em qualquer lugar do mundo ou com qualquer pessoa, como encontrava quando tomava café na mesa daquela cozinha; enquanto conversava com seu pai e sua mãe enchia a casa com cheiro de bolo no forno.

Dias felizes haviam passado, não tinha mais ninguém que dependesse dela, nem niguém com quem pudesse conversar, com quem pudesse dividir o pouco que sabia como dividir. Seus pais nunca souberam exatamente como ela ganhava dinheiro, mas partilhavam de boa parte de suas conquistas e de suas dores. Sozinha ela tinha aprendido a se apresentar Francisca ao mundo; sozinha tinha sofrido as dores de suas escolhas. Mas sempre tinha um final de semana em que tirava o peso do mundo de suas costas naquela cozinha, em que podiam conversar amavelmente sobre João Pedro e suas travessuras cândidas, sobre as desventuras e conquistas de Francisca, sobre a saúde de seu pai e os afazeres de sua mãe.

Esses finais de semana não existiriam mais. Já não tinha também o conforto esporádico e fugaz da clientela que cultivou na cidade. Não tinha nem mais um fio de amor no mundo.

Algumas semanas depois do enterro de sua mãe, Joana recebeu uma ligação de uma das vizinhas de seus pais que, com alguma apatia, a comunicava que a polícia havia encontrado sua irmã morta em casa.

Francisca deixou o mundo sozinha e por escolha sua. Exatamente da mesma forma que tinha vivido.

\end{document}
