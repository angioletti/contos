\documentclass[11pt,a4paper,twoside,openany]{book}
\usepackage[utf8]{inputenc}
\usepackage[T1]{fontenc}
\usepackage{fullpage}

\begin{document}
\chapter*{Quando a conheci}

"Ela está perdendo pontinhos por atraso, já". Eram 20h30, e eu escrevia para uma amiga enquanto esperava pela Carol já há meia hora.

Assim que levantei os olhos do celular vi uma pessoa andando na calçada, a caminho do restaurante em que eu estava; o cabelo parecia o dela, o tempo parecia o dela (vinha de ônibus e tinha dito que chegaria em dez minutos, dez minutos atrás); mas não podia ter certeza, ainda não a conhecia.

Daqueles \emph{matches} (na verdade um \emph{crush}, como ela me corrigiria em algum ponto da noite) que se faz virtualmente, estávamos nos conhecendo pela primeira vez, depois de termos trocado muitas palavras online por uma semana. A conversa era fácil, fazíamos um ao outro rir por texto, mesmo sem querer.

Uma mensagem no meu celular e eu sabia que era ela. "Onde você está?" Eu levantei e a vi, na entrada.

Ela veio na minha direção, nos abraçamos, eu caprichei no aperto, meio que prendendo ela um pouco a mais do que ela esperava.

Nos sentamos num banco de três lugares --- estávamos na área de espera do restaurante, um ambiente informal, com mesas improvisadas e bancos externos; um chafariz, uma leve chuva e as luzes da Paulista davam um charme rico ao ambiente e decidimos ficar por ali. Ótimo, o banco proporcionava uma proximidade maior, sem uma mesa entre nós.

Ela se sentou --- se jogou no banco, com seus cabelos soltos, seu decote e seu corpinho delicado de quase 1,60. Eu, todo preocupado em estar charmoso, percebo uma moça que senta de pernas abertas e braços jogados, na minha frente, falando alguma coisa rápida, de um assunto mal-começado, sem olhar pra mim. Por um momento eu achei que ela estivesse muito desconfortável em estar ali. Talvez eu não fosse o que ela esperava e ela pensava em ir embora. Toda a linguagem corporal dela dizia que ela, por algum motivo, não queria estar ali.

Mas sorria, e se dedicava na conversa, e achava graça dos meus comentários. Respondia minhas perguntas com franqueza. Pedimos um vinho, não comeríamos nada.

Com um pouco de tempo e de conversa ininterrupta, sentado ao lado dela, meu pescoço começou a doer, resolvi abdicar da posição de maior contato físico para a posição de maior contato no olhar; ocupei uma cadeira à frente dela.

O tempo inteiro, conversa ininterrupta, vinho, risadas e risadas, alguns olhares, palavras e mais vinho. Nos beijamos, muito bem, por sinal.

Ela levanta para ir ao banheiro e aproveito para ver o celular. Havíam algumas mensagens da minha amiga, preocupada com o bolo que eu tomaria, ao que respondi, simplesmente: o beijo compensou o atraso; e o resto da noite foi nosso, não menos divertido do que tinha sido até ali.

\end{document}
